%--------------------------------------------------------------------------
% Dokumentenklasse
%--------------------------------------------------------------------------

% disable Warning for remreset Package
% \RequirePackage{silence}
% \WarningFilter{remreset}{The remreset package}

\documentclass[
	pagesize,
	fontsize=12pt,
	paper=a4,
	oneside,
    reqno
]{scrartcl}

%--------------------------------------------------------------------------
% Standardpakete 
%--------------------------------------------------------------------------
\usepackage[ngerman]{babel}               % Deutsch Silbentrennung
\usepackage{csquotes}                     % Setzen von Zitaten
\usepackage{xspace}                       % setzten von Leerzeichen nach Abkürzungen
\usepackage{microtype}                    % für glattere Seitenränder
\renewcommand*\familydefault{\sfdefault}  % Serifen lose Schrift
%\renewcommand*\familydefault{\ttdefault} % Schreibmaschinenschrift

%--------------------------------------------------------------------------
% Extra Packages
%--------------------------------------------------------------------------

% Abkürzungspaket
\usepackage[]{acronym}
% \usepackage[nohyperlinks]{acronym} % Abkürzungsverzeichnis ohne Klick-Referenz zu den Abkürzungen im Text

% Mathe Pakete
\usepackage{amsmath}
\DeclareMathOperator{\sgn}{sgn}
\usepackage{thmtools}
\usepackage{amsfonts}
\usepackage{amssymb}
\usepackage{mathtools}
\usepackage{breqn}
\usepackage{empheq}
\numberwithin{equation}{section} % Nummerierung von Gleichungen innerhalb eines Kapitels

% Listenumgebungen
\usepackage{listings}
\usepackage{paralist}
\usepackage{enumitem}
\usepackage{adjustbox}

% Demo Text
\usepackage{blindtext}

% Farb-Pakete
\usepackage{xcolor}
\usepackage{fancyvrb}
\usepackage{colortbl}

% Farbedefinitionen
\definecolor{htw}{RGB}{120, 184, 2}
\definecolor{ccW}{RGB}{255,255,255}
\definecolor{ccR}{RGB}{197,14,31}
\definecolor{ccG}{RGB}{113,113,113}
\definecolor{ccL}{RGB}{220,220,220}
\definecolor{ccS}{RGB}{0,0,0}
\definecolor{ccB}{RGB}{68,73,159}
\definecolor{ccD}{RGB}{0,0,80}
\definecolor{grey}{RGB}{200,200,200}
\definecolor{lightGrey}{RGB}{230,230,230}
\definecolor{darkgrey}{RGB}{120,120,120}
\definecolor{orange}{RGB}{240,120,0}

% Für erweiterte Tabellen
\usepackage{longtable}
\usepackage{tabularx}
\usepackage{float}
\usepackage{multirow}
\usepackage{makecell}
\numberwithin{table}{section} % Nummerierung von Tabellen innerhalb eines Kapitels
% \setlength{\tabcolsep}{0.5em}       % for the horizontal padding
% {\renewcommand{\arraystretch}{1.8}  % for the vertical padding
% \usepackage{ragged2e}
% \newcolumntype{R}[1]{>{\RaggedRight}p{#1}}

% Einheitenpaket
\usepackage[exponent-product = \cdot]{siunitx}
\sisetup{locale=DE}
\sisetup{parse-numbers = false}

\makeatletter
\renewcommand\@dotsep{5}
\makeatother

% Pakete für Grafiken
\usepackage{graphicx}
\usepackage{wrapfig}
\usepackage{overpic}
\usepackage{epstopdf}
\usepackage{caption}
\usepackage{subcaption}
\usepackage{rotating}
\usepackage{lscape}
\numberwithin{figure}{section} % Bildnummerierung Kapitelweise
% \captionsetup[subfigure]{list=true, font=normalsize, labelformat=brace, position=top} %setup für subfigure captions

% Diagramm-/Grafikerstellung
\usepackage{pstricks}
\usepackage{pst-node}
\usepackage{pst-coil}
\usepackage{tikz}
\usetikzlibrary{patterns,patterns.meta}
\makeatletter
\pgfkeys{/pgf/.cd,
  parallelepiped offset x/.initial=2mm,
  parallelepiped offset y/.initial=2mm
}
\pgfdeclareshape{parallelepiped}
{
  \inheritsavedanchors[from=rectangle] % this is nearly a rectangle
  \inheritanchorborder[from=rectangle]
  \inheritanchor[from=rectangle]{north}
  \inheritanchor[from=rectangle]{north west}
  \inheritanchor[from=rectangle]{north east}
  \inheritanchor[from=rectangle]{center}
  \inheritanchor[from=rectangle]{west}
  \inheritanchor[from=rectangle]{east}
  \inheritanchor[from=rectangle]{mid}
  \inheritanchor[from=rectangle]{mid west}
  \inheritanchor[from=rectangle]{mid east}
  \inheritanchor[from=rectangle]{base}
  \inheritanchor[from=rectangle]{base west}
  \inheritanchor[from=rectangle]{base east}
  \inheritanchor[from=rectangle]{south}
  \inheritanchor[from=rectangle]{south west}
  \inheritanchor[from=rectangle]{south east}
  \backgroundpath{
    % store lower right in xa/ya and upper right in xb/yb
    \southwest \pgf@xa=\pgf@x \pgf@ya=\pgf@y
    \northeast \pgf@xb=\pgf@x \pgf@yb=\pgf@y
    \pgfmathsetlength\pgfutil@tempdima{\pgfkeysvalueof{/pgf/parallelepiped offset x}}
    \pgfmathsetlength\pgfutil@tempdimb{\pgfkeysvalueof{/pgf/parallelepiped offset y}}
    \def\ppd@offset{\pgfpoint{\pgfutil@tempdima}{\pgfutil@tempdimb}}
    \pgfpathmoveto{\pgfqpoint{\pgf@xa}{\pgf@ya}}
    \pgfpathlineto{\pgfqpoint{\pgf@xb}{\pgf@ya}}
    \pgfpathlineto{\pgfqpoint{\pgf@xb}{\pgf@yb}}
    \pgfpathlineto{\pgfqpoint{\pgf@xa}{\pgf@yb}}
    \pgfpathclose
    \pgfpathmoveto{\pgfqpoint{\pgf@xb}{\pgf@ya}}
    \pgfpathlineto{\pgfpointadd{\pgfpoint{\pgf@xb}{\pgf@ya}}{\ppd@offset}}
    \pgfpathlineto{\pgfpointadd{\pgfpoint{\pgf@xb}{\pgf@yb}}{\ppd@offset}}
    \pgfpathlineto{\pgfpointadd{\pgfpoint{\pgf@xa}{\pgf@yb}}{\ppd@offset}}
    \pgfpathlineto{\pgfqpoint{\pgf@xa}{\pgf@yb}}
    \pgfpathmoveto{\pgfqpoint{\pgf@xb}{\pgf@yb}}
    \pgfpathlineto{\pgfpointadd{\pgfpoint{\pgf@xb}{\pgf@yb}}{\ppd@offset}}
  }
}
\makeatother
\usetikzlibrary{math}
\usepackage{pgfplots}
\pgfplotsset{compat=1.5}
\usetikzlibrary{intersections,positioning,arrows,automata,calc,patterns,shapes.multipart,fit,backgrounds,decorations.pathreplacing}
\usetikzlibrary{decorations,shapes.geometric}
\usetikzlibrary{matrix,calc,angles,positioning,quotes}
% \usepackage{tikz-uml}

\usepackage{pgfkeys}
\usepackage{pgfopts}
\usepackage{ifthen}
\usepackage{xstring}
\usepackage{calc}
\usepackage{pst-plot,pst-bar,pst-node} % Balkendiagramme
\usepackage{capt-of}
\usepackage{incgraph} % Fullscreen Images
\usepackage{pdfpages} % Include external pdf pages

\usepackage{latexsym}
\usepackage{censor}
\usepackage{here}
% \StopCensoring        % Auskommentiert wird der Text entschwaerzt 
% \censor{Oszilloskop}  % Befehl zum einschwärzen
\usepackage{trfsigns}   % Transformation Symbol o---o \laplace and \Laplace
\usepackage{circuitikz}

\usepackage{multido}

% Verlinkungen im Text
\usepackage{url}
\usepackage{hyperref}
\PassOptionsToPackage{hyphens}{url}
\hypersetup{hidelinks}
\urlstyle{same}

%--------------------------------------------------------------------------
% Eigene Befehle
%--------------------------------------------------------------------------
\newcommand*\widefbox[1]{\fbox{\hspace{1em}#1\hspace{1em}}} % Box für align-Umgebung

%------------sectioning command-------------------
% The sectioning command one level down the hierarchy from \subsubsection is called \paragraph followed by \subparagraph
% to include this in your table of contents

% Tiefe des Inhaltsverzeichnis
\setcounter{tocdepth}{2}
\setcounter{secnumdepth}{4}

% Abkürzungen durch Kommandos setzen
\newcommand{\bspw}{bspw.\xspace}
\newcommand{\bzw}{bzw.\xspace}
\newcommand{\etc}{etc.\xspace}
\newcommand{\zB}{z.\,B.\xspace}
\newcommand{\EV}{e.\,V.\xspace}
\newcommand{\zT}{z.\,T.\xspace}
\newcommand{\iVm}{i.\,V.\,m.\xspace}
\newcommand{\idR}{i.\,d.\,R.\xspace}
\newcommand{\ihv}{i.\,H.\,v.\xspace}
\newcommand{\ua}{u.\,a.\xspace}
\newcommand{\dH}{d.\,h.\xspace}
\newcommand{\vgl}{vgl.\xspace}
\newcommand{\ca}{ca.\xspace}
\newcommand{\dV}{d.\,Verf.}
\newcommand{\RNr}{Rn.\xspace}
\newcommand{\oa}{o.\,{ä}.\xspace}
\newcommand{\vC}{v.\,Chr.\xspace}
\newcommand{\nC}{n.\,Chr.\xspace}
\newcommand{\vA}{v.\,a.\xspace}
\newcommand{\eng}{engl.\xspace}
\newcommand{\tabitem}{~~\llap{\textbullet}~~}

%------------Zitate-------------------------------
\newcommand*{\zitat}[2]{%
   \normalfont\small
   \begin{quote}
   \glqq#1\grqq \par
   #2
   \end{quote}
   \normalsize
}
\newcommand*{\zitatmitueberschrift}[3]{%
   \normalfont\small
   \begin{quote} #3
   \glqq#1\grqq \par
   #2
   \end{quote}
   \normalsize
}
\newcommand*{\zitext}[2]{%
   \glqq#1\grqq\ %
   [#2]%
}

%-----------Seitendesign--------------------------
\usepackage[width=15.5cm, height=23cm, includeheadfoot]{geometry}
\geometry{paper=a4paper}
% \usepackage[left=6cm,right=1cm,top=1.5cm, bottom=1cm, includeheadfoot]{geometry}
% \newgeometry{oneside}
% \setlength{\voffset}{0cm}
\setlength{\headheight}{1.1\baselineskip} % increase headheight
\setlength{\footheight}{28.99998pt}       % increase foodheight
\setlength{\parindent}{0cm}               % Einrücken nach \newline
\setlength{\footskip}{86pt}               % Move Footer down
% \setlength{\topmargin}{0cm}
% \setlength{\marginparsep}{0.5cm}
% \setlength{\marginparwidth}{1.5cm}
% \setlength{\textwidth}{16cm}
% \setlength{\textheight}{23cm}
% \setlength{\oddsidemargin}{1cm}
% \setlength{\evensidemargin}{2cm}

%----------Kopf & Fußzeile------------------------
% \usepackage[headsepline,footsepline]{scrpage2}
\usepackage[headsepline]{scrlayer-scrpage}
\pagestyle{scrheadings}
\clearpairofpagestyles
\ihead{\headmark}
\automark{section}
\chead{}
\ohead{\includegraphics[scale=0.09]{Bilder/HTWLogoKopfzeile.png} \nocite{HTWklein}}
\ifoot{Christopher Berg\\ Sebastian Richter\\ Aaron Zielstorff}
\cfoot{\pagemark}
\ofoot{VA3 Automation in\\ regenerativen Energiesystemen}

%--------------------------------------------------------------------------
% Beginn des Dokuments
%--------------------------------------------------------------------------
\begin{document}

%----------Deckblatt----------------------------- 
\begin{titlepage}
   \pagestyle{empty} % setzt Pagestyle-Befehl

   % HTW Logo
   \begin{flushright}
   \includegraphics[scale=.07]{Bilder/LogoHTWBerlin.png}  \nocite{HTWgross}
   \end{flushright}

   \vspace{1cm}

   % Titel
   \begin{center}
      \Huge{\textbf{Modellierung, Simulation und Regelung einer drehzahlvariablen Windturbine}} \\
   \end{center}
   
   \vspace{0.5cm}
   
   % Model
   \begin{center}
      \Large{\textbf{Automation in regenerativen Energiesystemen (VA3)}} \\
   \end{center}

   \vspace{3cm}

   % Name
   \begin{flushleft}
      \begin{tabular}{l c l }
         \textbf{Name: }&\hspace{1 cm} &\textbf{Matrikelnummer:} \\
         Christopher Berg   & & 579665 \\
         Sebastian Richter  & & 572906 \\
         Aaron Zielstorff   & & 567183 \\
      \end{tabular}
   \end{flushleft}

   \vspace{1cm}

   % Daten
   \begin{tabular}{l l}
      \textbf{Fachbereich:}   & FB1                                                 \\
      \textbf{Studiengang:}   & M.\xspace Elektrotechnik                            \\
      \textbf{Fachsemester:}  & 3.\xspace FS                                        \\
      \textbf{Fach:}          & VA3 Automation in regenerativen Energiesystemen     \\
      \textbf{Dozent:}        & Prof.\xspace Dr.\xspace -Ing.\xspace Horst Schulte  \\
      \textbf{Abgabe am:}     & 10.\xspace Februar 2022                             \\ 
   \end{tabular}
\end{titlepage}
\clearpage

%--------Inhaltsverzeichnis-----------------------
\renewcommand{\contentsname}{Inhaltsverzeichnis}
\tableofcontents
\clearpage

%--------Abbildungsverzeichnis--------------------
\renewcommand{\listfigurename}{Abbildungsverzeichnis}
\renewcommand*{\figurename}{Abb.}
\listoffigures
% \clearpage

%--------Tabellenverzeichnis----------------------
\renewcommand*{\listtablename}{Tabellenverzeichnis}
\renewcommand*{\tablename}{Tab.}
\listoftables
%\clearpage

%--------------Symbolverzeichnis------------------
\section*{Symbolverzeichnis}
\begin{acronym}[Symbols]
    \acro{A}        [$A$]                       {Querschnittsfläche}
    \acro{Ai}       [$A_{\mathrm{i}}$]          {Querschnittfläche an Stelle i}    
    \acro{A1}       [$A_{\mathrm{1}}$]          {Eintritts- (kontroll-) Fläche einer Strömungsröhre}
    \acro{A2}       [$A_{\mathrm{2}}$]          {Rotorfläche}
    \acro{A3}       [$A_{\mathrm{3}}$]          {Austritts- (kontroll-) Fläche einer Strömungsröhre}
    \acro{c}        [$c$]                       {Anströmgeschwindigkeit}
    \acro{cA}       [$c_{\mathrm{A}}$]          {Autriebsbeiwert}
    \acro{cW}       [$c_{\mathrm{W}}$]          {Widerstandbeiwert}
    \acro{cM}       [$c_{\mathrm{M}}$]          {Momentenbeiwert}
    \acro{cMopt}    [$c_{\mathrm{M,opt}}$]      {optimaler Momentenbeiwert}
    \acro{cP}       [$c_{\mathrm{P}}$]          {Leistungsbeiwert}
    \acro{cPopt}    [$c_{\mathrm{P,opt}}$]      {optimaler Leistungsbeiwert}
    \acro{cPmax}    [$c_{\mathrm{P,max}}$]      {max. Leistungsbeiwert}
    \acro{cs}       [$c_{\mathrm{s}}$]          {Schubbeiwert}
    \acro{EW}       [$E_{\mathrm{W}}$]          {Kinetische Energie des Windes}
    \acro{FA}       [$F_{\mathrm{A}}$]          {Auftriebskraft (am Flügel-/Blattprofil)}
    \acro{Fres}     [$F_{\mathrm{res}}$]        {resultierende Kraft}
    \acro{deltaFU}  [$\Delta F_{\mathrm{U}}$]   {anteilige Umfangskraft}
    \acro{deltaFS}  [$\Delta F_{\mathrm{S}}$]   {anteilige Schubkraft}
    \acro{FS}       [$F_{\mathrm{S}}$]          {Schubkraft auf den Rotor}
    \acro{FSA}      [$F_{\mathrm{S,A}}$]        {Schubkraftanteil der Auftriebskraft}
    \acro{FSW}      [$F_{\mathrm{S,W}}$]        {Schubkraftanteil der Widerstandskraft}
    \acro{FST}      [$F_{\mathrm{ST}}$]         {Staukraft am Rotor}
    \acro{FUA}      [$F_{\mathrm{U,A}}$]        {Umfangskraftanteil der Auftriebskraft}
    \acro{FUW}      [$F_{\mathrm{U,W}}$]        {Umfangskraftanteil der Widerstandskraft}
    \acro{FW}       [$F_{\mathrm{W}}$]          {Widerstandskraft (am Flügel-/Blattprofil)}
    \acro{JB}       [$J_{\mathrm{B}}$]          {Massenträgheitsmoment eines Rotorblattes}
    \acro{JG}       [$J_{\mathrm{G}}$]          {Massenträgheitsmoment des Generators}
    \acro{JLW}      [$J_{\mathrm{LW}}$]         {Massenträgheitsmoment der Langsamen (Haupt-) Welle}
    \acro{JN}       [$J_{\mathrm{N}}$]          {Massenträgheitsmoment der Nabe}
    \acro{JR}       [$J_{\mathrm{R}}$]          {Massenträgheitsmoment des Rotors (= Nabe + Blätter)}
    \acro{JSW}      [$J_{\mathrm{SW}}$]         {Massenträgheitsmoment der schnellen (Zwischen-) Welle}
    \acro{MG}       [$M_{\mathrm{G}}$]          {Generatordrehmoment}
    \acro{MGnenn}   [$M_{\mathrm{G,nenn}}$]     {Generatordrehnennmoment}
    \acro{PG}       [$P_{\mathrm{G,nenn}}$]     {Generatornennleistung}
    \acro{nG}       [$n_{\mathrm{G,nenn}}$]     {Generatornenndrehzahl}
    \acro{MR}       [$M_{\mathrm{R}}$]          {Rotordrehmoment}
    \acro{MW}       [$M_{\mathrm{W}}$]          {\glqq Wind\grqq{}-drehmoment}
    \acro{MB}       [$M_{\mathrm{B}}$]          {Blattdrehmoment}
    \acro{mNac}     [$m_{\mathrm{Nac}}$]        {Gondelmasse}
    \acro{mRot}     [$m_{\mathrm{Rot}}$]        {Rotormasse (= Nabe + Blätter)}
    \acro{mTow}     [$m_{\mathrm{Tow}}$]        {Turmmasse}
    \acro{mT}       [$m_{\mathrm{T}}$]          {Ersatzmasse der gesamten Windkraftanlage}
    \acro{mBla}     [$m_{\mathrm{Bla}}$]        {Masse eines Rotorblattes}
    \acro{mB}       [$m_{\mathrm{B}}$]          {Effektiv schwingende Blattmasse}
    \acro{ks}       [$k_{\mathrm{s}}$]          {Triebsstrangsteifigkeit bezogen auf die schnelle Welle}
    \acro{kT}       [$k_{\mathrm{T}}$]          {Ersatzsteifigkeit des Turmes}
    \acro{kB}       [$k_{\mathrm{B}}$]          {Ersatzsteifigkeit eines Blattes}
    \acro{ds}       [$d_{\mathrm{s}}$]          {Dämpfungsfaktor des Triebsstranges bezogen auf die schnelle Welle}
    \acro{dT}       [$d_{\mathrm{T}}$]          {Dämpfungsfaktor des Turmes}
    \acro{dB}       [$d_{\mathrm{B}}$]          {Dämpfungsfaktor eines Blattes}
    \acro{yT}       [$y_{\mathrm{T}}$]          {Turmverbiegung}
    \acro{yB}       [$y_{\mathrm{B}}$]          {Blattverbiegung}
    \acro{m}        [$m$]                       {Luftmasse}
    \acro{mi}       [$m_{\mathrm{i}}$]          {Luftmasse an Stelle i}
    \acro{mdot}     [$\dot m$]                  {Luftmassenstrom}
    \acro{mdoti}    [$\dot{m}_{\mathrm{i}}$]    {Luftmassenstrom an Stelle i}
    \acro{mdot1}    [$\dot{m}_{\mathrm{1}}$]    {Luftmassenstrom an Stelle 1}
    \acro{mdot2}    [$\dot{m}_{\mathrm{2}}$]    {Luftmassenstrom an Stelle 2}
    \acro{mdot3}    [$\dot{m}_{\mathrm{3}}$]    {Luftmassenstrom an Stelle 3}
    \acro{ng}       [$n_{\mathrm{g}}$]          {Getriebeübersetzung}
    \acro{nR}       [$n_{\mathrm{R}}$]          {Rotordrehzahl}
    \acro{PR}       [$P_{\mathrm{R}}$]          {Mechanische Leistung des Rotors}
    \acro{PW}       [$P_{\mathrm{W}}$]          {Mechanische Leistung des Windes}
    \acro{r}        [$r$]                       {Effektive Blattlänge}
    \acro{ri}       [$r_{\mathrm{i}}$]          {Radius an Stelle i}
    \acro{R}        [$R$]                       {Rotoraußenradius}
    \acro{u}        [$u$]                       {Bewegungsgeschwindigkeit}
    \acro{u'}       [$u^{'}$]                   {relative Bewegungsgeschwindigkeit}
    \acro{v}        [$v$]                       {Hauptwindgeschwindigkeitsrichtung des Windvektors $\underline v$}
    \acro{vi}       [$v_{\mathrm{i}}$]          {Windgeschwindigkeit an Stelle i}
    \acro{vdoti}    [$\dot{v}_{\mathrm{i}}$]    {Windgeschwindigkeitsänderung an Stelle i}
    \acro{v1}       [$v_{\mathrm{1}}$]          {Windgeschwindigkeit vor Rotorebene}
    \acro{v2}       [$v_{\mathrm{2}}$]          {Windgeschwindigkeit in der Rotorebene}
    \acro{v3}       [$v_{\mathrm{3}}$]          {Windgeschwindigkeit hinter der Rotorebene}
    \acro{z}        [$z$]                       {Anzahl der Rotorblätter eines Rotors}
    \acro{ThetaP}   [$\Theta_{\mathrm{P}}$]     {Pitchwinkel}
    \acro{lambda}   [$\lambda$]                 {Schnelllaufzahl}
    \acro{lambdaopt}[$\lambda_{\mathrm{opt}}$]  {optimale Schnelllaufzahl}
    \acro{rho}      [$\rho$]                    {Luftdichte}
    \acro{rhoi}     [$\rho_{\mathrm{i}}$]       {Luftdichte an Volumenstelle i}
    \acro{rho1}     [$\rho_{\mathrm{1}}$]       {Luftdichte an Volumenstelle 1}
    \acro{rho2}     [$\rho_{\mathrm{2}}$]       {Luftdichte an Volumenstelle 2}
    \acro{rho3}     [$\rho_{\mathrm{3}}$]       {Luftdichte an Volumenstelle 3}
    \acro{omega}    [$\omega$]                  {Winkelgeschwindigkeit}
    \acro{omegaG}   [$\omega_{\mathrm{G}}$]     {Generatorwinkelgeschwindigkeit}
    \acro{omegaR}   [$\omega_{\mathrm{R}}$]     {Ro\-tor\-win\-kel\-ge\-schwin\-dig\-keit}
    \acro{omegadot} [$\dot \omega$]             {Winkelbeschleunigung}
    \acro{phiG}     [$\varphi_{\mathrm{G}}$]    {Generator Winkel}
    \acro{dotphiG}  [$\dot\varphi_{\mathrm{G}}$]{Generator Winkelgeschwindigkeit}
    \acro{phiR}     [$\varphi_{\mathrm{R}}$]    {Rotor Winkel}
    \acro{dotphiR}  [$\dot\varphi_{\mathrm{R}}$]{Rotor Winkelgeschwindigkeit}
    \acro{phiRtilde}[$\tilde{\varphi}_{\mathrm{R}}$]    {Rotor Winkel bezogen auf die Generatorseite des Getriebes}
    \acro{pi}       [$\pi$]                     {Kreiszahl}
    \acro{pST}      [$p_{\mathrm{ST}}$]         {Staudruck}
    \acro{cT}       [$c_{\mathrm{T}}$]          {Schubkraftbeiwert}
    \acro{alpha}    [$\alpha$]                  {Anstellwinkel}
    \acro{alphaopt} [$\alpha_{\mathrm{opt}}$]   {optimaler Anstellwinkel}
    \acro{tflug}    [$t_{\mathrm{Flug}}$]       {Tiefe eines Tragflügels}
    \acro{bflug}    [$t_{\mathrm{Flug}}$]       {Breite eines Tragflügels}
    \acro{eps}      [$\epsilon$]                {Gleitzahl}
    \acro{epsopt}   [$\epsilon_{\mathrm{opt}}$] {optimale Gleitzahl}
    \acro{gamma}    [$\gamma$]                  {Anströmwinkel}
    \acro{beta}     [$\beta$]                   {Bauwinkel}
    \acro{kI}       [$k_{\mathrm{I}}$]          {Faktor für die Steuerung im unteren Teillastbereich, konstant}
    \acro{komegaR}  [$k_{\omega\mathrm{R}}$]   {Linearisierungkoeffizient}
    \acro{ktheta}   [$k_{\theta}$]     {Linearisierungkoeffizient}
    \acro{kv}       [$k_{\mathrm{v}}$]          {Linearisierungkoeffizient}
    \end{acronym}
\clearpage

%---------Kapitel/Text----------------------------

\section{Einführung in die Windenergieanlage} \label{einfuehrung}
% Aaron

Ziel dieser Arbeit soll es sein, eine drehzahlvariable \SI{5}{MW} Windturbine zu Modellieren, Simulieren und die Regelung umzusetzen. Konkret handelt es sich um eine \textit{NREL}-Turbine, die für den Offshore-Einsatz konzipiert ist.\\
Dafür sollen folgende Anforderungen umgesetzt werden:

\begin{enumerate}
    \item Erstellung des mathematischen Modells der Windturbine
    \item Implementierung des Modells in Matlab/Simulink
    \item Untergliederung des Modells in die Teilmodelle \texttt{Antriebsstrang}, \texttt{Aerodynamik}, \texttt{Turm- und Blattdynamik}
    \item Umsetzung eines reduzierten Windturbinen-Modells für den Teil- und Volllastbereich
    \item Reglerentwurf für alle Arbeitspunkte (über kennfeldbasierte, arbeitspunktabhängige Nachführung der Reglerkoeffizienten)
\end{enumerate}

Der modellhafte Aufbau einer Windturbine ist nachfolgend (in \autoref{fig:Bild1.1}) dargestellt.

\begin{figure}[H]
   \centering
   \begin{pspicture}[showgrid=false](0,0)(14.6,5)
        \psframe(0,0)(14.6,5)
        % Rotor
        \pscircle(2.7,2.5){0.07}
        
        \psline(2.4,2.5)(2.63,2.5)
        \psline(2.46,2.42)(2.46,2.58)
        \pspolygon(2.4,2.58)(2.4,2.42)(0.6,2.48)(0.6,2.52)
        
        \psline(2.73,2.55)(2.86,2.75)
        \psline(2.76,2.75)(2.89,2.66)
        \pspolygon(2.78,2.8)(2.93,2.7)(3.88,4.26)(3.84,4.28)
        
        \psline(2.73,2.45)(2.86,2.25)
        \psline(2.76,2.25)(2.89,2.34)
        \pspolygon(2.78,2.2)(2.93,2.3)(3.88,0.74)(3.84,0.72)
        
        % Verbindung Rotor -> Getriebe
        \psline[linewidth=2pt](2.77,2.5)(6,2.5)
        
        % Getriebe
        \psframe(5.77,1.73)(6.97,2.93)
        \pscircle[linestyle=dashed](6.3,2.5){0.3}
        \psdot(6.3,2.5)
        \pscircle[linestyle=dashed](6.55,2.05){0.2}
        \psdot(6.55,2.05)
        \rput(6.35,3.1){\tiny Getriebe}
        
        % Verbindung Getriebe -> SG
        \psline[linewidth=2pt](6.75,2.05)(8,2.05)
        
        % SG
        \pscircle(8.3,2.05){0.3}
        \pscircle(8.3,2.05){0.37}
        \rput(8.3,2.05){\small SG}
        
        % Verbindung SG -> Umrichter
        \psline(8.67,2.05)(9.5,2.05)
        \psline(9,2.15)(8.8,1.95)
        \psline(9.1,2.15)(8.9,1.95)
        \psline(9.2,2.15)(9,1.95)
        
        % Umrichter
        \psframe[linestyle=dashed,linecolor=darkgrey](9.4,1.5)(11.8,3)
        \rput(10.6,1.3){\tiny Umrichter}
        
        \psframe(9.5,1.65)(10.3,2.45)
        \psline(9.52,1.67)(10.28,2.43)
        \rput(9.9,2.7){\tiny GR}
        \rput(9.77,2.25){\tiny $3\sim$}
        \rput(10.07,1.85){\tiny $=$}
        
        \psline(10.3,2.05)(10.9,2.05)
        
        \psframe(10.9,1.65)(11.7,2.45)
        \psline(10.92,1.67)(11.68,2.43)
        \rput(11.3,2.7){\tiny WR}
        \rput(11.12,2.22){\tiny $=$}
        \rput(11.42,1.85){\tiny $3\sim$}
        
        % Trafo
        \psline(11.7,2.05)(12.3,2.05)
        \pscircle(12.6,2.05){0.3}
        \pscircle(13,2.05){0.3}
        \psline(13.3,2.05)(13.9,2.05)
        
        % Netzleitung
        \psline(13.9,0.6)(13.9,4.4)
        \psline(13.8,3)(14,3.2)
        \psline(13.8,2.9)(14,3.1)
        \psline(13.8,2.8)(14,3)
        \psdot(13.9,2.05)
    \end{pspicture}
   \caption[Aufbau NREL Windturbine]{Modellhafte Darstellung einer NREL Windturbine}
   \label{fig:Bild1.1}
\end{figure} % Modelldarstellung NREL Windturbine

Wie bereits aus den Anforderungen hervorgeht, soll das umzusetzende Modell unterteilt werden. Dabei besitzt jedes Teilmodell eigene Parameter/Konstanten, die in \autoref{tab:Tabelle1.1} aufgezeigt sind.

\begin{table}[H]
    \centering
    \begin{tabular}{|lll|}
        \hline
        \rowcolor{grey}
        \textbf{Symbol}          & \textbf{Parameter}                               & \textbf{Wert}                                                         \\ \hline
        \rowcolor{lightGrey}
        \multicolumn{3}{|c|}{Antriebsstrang}                                                                                                                \\ \hline
        \acs{ng}                 & Getriebeübersetzungsverhältnis                   & $97.0$                                                                \\
        \acs{JR}                 & Rotor Trägheitsmoment                            & $\SI{38759 \cdot 10^{3}}{kg \cdot m^{2}}$                             \\
        \acs{JG}                 & Generator Trägheitsmoment                        & $\SI{534.1}{kg \cdot m^{2}}$                                          \\
        \acs{ks}                 & Triebsstrangsteifigkeit bez. auf schnelle Welle  & $867637000 / n_{\mathrm{g}}^{2}$                                      \\
        \acs{ds}                 & Dämpfungsfaktor d. Triebsstranges                & $6215000 / n_{\mathrm{g}}^{2}$                                        \\ \hline
        \rowcolor{lightGrey}
        \multicolumn{3}{|c|}{Turm}                                                                                                                          \\ \hline
        \acs{mNac}               & Gondelmasse                                      & $\SI{240000}{kg}$                                                     \\
        \acs{mRot}               & Rotormasse (Blätter und Narbe)                   & $\SI{11000}{kg}$                                                      \\
        \acs{mTow}               & Turmmasse                                        & $\SI{347460}{kg}$                                                     \\
        \acs{mT}                 & Ersatzmasse der Windkraftanlage                  & $m_{\mathrm{Nac}} + m_{\mathrm{Rot}} + 0.25 \cdot m_{\mathrm{Tow}}$   \\
        \acs{kT}                 & Ersatzsteifigkeit des Turmes                     & $\SI{1981900}{N / m}$                                                 \\
        \acs{dT}                 & Dämpfungsfaktor des Turmes                       & $7 \cdot 10^{4}$                                                      \\ \hline
        \rowcolor{lightGrey}
        \multicolumn{3}{|c|}{Rotorblatt}                                                                                                                    \\ \hline
        \acs{R}                  & Rotoraußenradius                                 & $\SI{63}{m}$                                                          \\
        \acs{mBla}               & Masse eines Rotorblattes                         & $\SI{17740}{kg}$                                                      \\
        \acs{r}                  & Effektive Blattlänge                             & $\SI{21.975}{m}$                                                      \\
        \acs{mB}                 & Effektiv schwingende Blattmasse                  & $0.25 \cdot m_{\mathrm{Bla}}$                                         \\
        \acs{kB}                 & Ersatzsteifigkeit eines Blattes                  & $\SI{40000}{N / m}$                                                   \\
        \acs{dB}                 & Dämpfungsfaktor eines Blattes                    & $2 \cdot 10^{4}$
                                \\ \hline
        \rowcolor{lightGrey}
        \multicolumn{3}{|c|}{Weitere Parameter}                                                                                                             \\ \hline
        \acs{rho}                & Luftdichte                                       & $\SI{1.225}{kg / m^{3}}$                                              \\ \hline
    \end{tabular}
    \caption[Modellparameter]{Modellparameter der NREL Windturbine}
    \label{tab:Tabelle1.1}
\end{table}

Ziel soll es sein eine Regelung für den Teillast- und Volllastbetrieb umzusetzen, die auf die in Simulink implementierten (Teil-)Modelle angewendet wird. Als Stellgröße gelten das Generatormoment und der kollektive Pitchwinkel. Dabei sind folgende Systemgrenzen zu berücksichtigen:

\begin{enumerate}
    \item Stellgrößenbegrenzung des Pitchantriebes von maximal $8^\circ$
    \item Maximale Narbenauslenkung bei Böen sei $\SI{1.5}{m}$
    \item Maximale Blattauslenkung an der Spitze sei $\SI{7}{m}$
    \item Die Rotordrehzahl darf maximal 1.2-fach so groß sein wie die Nenndrehzahl
\end{enumerate}

\section{Theoretische Grundlagen} \label{theo_grundl}

\subsection{Stromröhrentheorie}

\subsection{Tragflügeltheorie}

\subsection{WEA-Kennfelder}

\subsection{Lookup Tables}


\section{Modellierung des Antriebsstranges} \label{modellierung_antriebsstrang}
% Aaron

\section{Momentenregelung des Antriebstrangs} \label{regelung}

\subsection{Unterer Teillastbereich}

\subsection{Oberer Teillastbereich}

\subsection{Volllastbereich}

\section{Turm- und Blatt-Modell} \label{turm_blatt}

\subsection{Aerodynamik}

\subsection{Modellierung des Turmes und des Blattes}

\begin{figure}[H]
\centering
\scalebox{1.0}{
\begin{tikzpicture}[
  scale = 1.0,thick,>=triangle 45,
  spring/.style = {decorate,decoration={zigzag,amplitude=2pt,segment length=4pt}}
  ]
  %Raster
\draw [color=gray!10] [step=1mm] (-2.5cm,1.1cm) grid (2.5cm,-7.5cm);
\draw [color=gray!30] [step=1cm] (-2.5cm,1.1cm) grid (2.5cm,-7.5cm);

%Blattmodell
  \node[draw,minimum size=1.5em] (mB) {$3\cdot m_B$};
  \draw[shorten <=3pt] (mB) -- coordinate (xB) ++( 2,0);
  \draw[shorten <=3pt] (mB) -- coordinate (f) ++(-2,0);
  \draw[<-] (f) -- node[left] {$F_T(t)$} ++(0,-1);
  \draw[->] (xB) -- node[right] {$x_B(t)$} ++(0,1);
  \draw (mB) -- +(0,-.7) coordinate (cB);
  \draw (cB) -| +( .7,-.3) coordinate (rB);
  \draw (cB) -| +(-.7,-.3) coordinate (lB);
  \draw (rB) -| +( .2,-1) node[right,pos=.75] {$d_B$};
  \draw (rB) -| +(-.2,-1);
  \draw (rB) +(-.15,-.5) -- coordinate (dB) +(.15,-.5);
  \draw[spring] (lB) -- node[left] {$k_B$} +(0,-1);
  \draw (lB) ++(0,-1) -- ++(0,-.3) coordinate (bB);
  \draw (dB) |- (bB);
  \draw (bB -| mB) -- ++(0,-.5) coordinate (gB);

%Turmmodell
	\node[coordinate] (mT) at (0.0,-4.0) {};
	\draw[shorten <=9pt] (mT) -- (gB);
	\node[draw,minimum size=1.5em] at (mT) {$m_T$};
  	\draw[shorten <=16pt] (mT) -- coordinate (xT) ++( 2,0);
  	\draw[shorten <=16pt] (mT) -- coordinate (f) ++(-2,0);
  	\draw[<-] (f) -- node[left] {$F_T(t)$} ++(0,-1);
  \draw[->] (xT) -- node[right] {$x_T(t)$} ++(0,1);
  \draw[shorten <=9.0pt] (mT) -- +(0,-.7) coordinate (c);
  \draw (c) -| +( .7,-.3) coordinate (r);
  \draw (c) -| +(-.7,-.3) coordinate (l);
  \draw (r) -| +( .2,-1) node[right,pos=.75] {$d_T$};
  \draw (r) -| +(-.2,-1);
  \draw (r) +(-.15,-.5) -- coordinate (d) +(.15,-.5);
  \draw[spring] (l) -- node[left] {$k_T$} +(0,-1);
  \draw (l) ++(0,-1) -- ++(0,-.3) coordinate (b);
  \draw (d) |- (b);
  \draw (b -| mT) -- ++(0,-.5) coordinate (g);
  \draw (g) +(-.5,0) -- +(.5,0);
  \foreach \i in {-1,0,1}
    \draw (g) ++(\i*.5,0) -- ++(225:.5);
 
\end{tikzpicture}
}
\caption{Modell des Turms und der Blätter}
\label{fig:TurmBlattModell}
\end{figure}

\section{Ausblick} \label{ausblick}

In dieser Ausarbeitung konnte erfolgreich gezeigt werden, wie eine Windenergieanlage modelliert und anschließend anhand des entwickelten Modells geregelt werden kann. Es hat sich jedoch auch gezeigt, dass einige Verbesserungen und Optimierungen von Nöten wären, bevor das Modell und insbesondere die Regelstruktur auf eine reale WEA angewendet werden kann. Nachfolgend sind die auffälligsten Mängel zusammengefasst mit möglichen Ansätzen für die Ausbesserung der Mängel \bzw die Optimierung des Verhaltens der Anlage.\\

Der Größte Mangel in der aktuellen Implementierung ist die Nichteinhaltung der gegebenen Constraints. Der Einsatz des implementierten Funktionsmodells würde voraussichtlich zur Zerstörung der realen Anlage sorgen. Als Lösungsansatz könnte der Detailgrad des Funktionsmodells angepasst werden, so dass als Grundlage für die Reglerimplementierung ein kombiniertes Rotor-/Turm-/Triebsstrangmodell dient.\\

Ebenfalls aufgefallen war das pendelnde Verhalten zwischen Betriebszuständen, wenn die maximale Windgeschwindigkeit in kurzen Zeitabständen überschritten wird. Hier wäre es vermutlich ratsam eine Zeitsperre in die Steuereinheit einzubauen, die dafür sorgt, dass die Anlage nicht sofort wieder anläuft. Es ist als wahrscheinlich einzustufen, dass wenn der Wind einmal die kritische Windgeschwindigkeit erreicht hat, dass er das auch weitere Male in der nächsten Zeit tut.\\

Die Reglervalidierung hat gezeigt, dass der obere Teillastbereich nur für einen sehr kleinen Bereich der Rotordrehzahl benötigt wird. Es ist schwer diesen Bereich dediziert anzufahren und damit auch nachzuweisen. Da der Bereich jedoch sehr klein ist, kann die Aussage getroffen werden, dass der vergleichsweise komplexe Regler auch mit einem Linearen Verhalten zwischen unterem- und Volllastbereich ersetzt werden könnte. Umgangssprachlich kann hier von einem over-Engineering geredet werden.\\

Als letzter Punkt soll an dieser Stelle noch die Relevanz eines Umrichters \bzw Umrichtermodells postuliert werden. Wie bereits in mehreren Stellen der Arbeit erwähnt, wurde lediglich das Störverhalten betrachtet. Die Implementierung eines Umrichters zur Leistungsoptimierung durch Drehzahlvariabilität würde die Umsetzung des Führungsverhaltens ermöglichen.

%---------Quellen---------------------------------
\newpage
\newcount\Quellennummer
\Quellennummer=1

\renewcommand\refname{Literaturverzeichnis}
\addcontentsline{toc}{section}{Literaturverzeichnis}

\begin{thebibliography}{999}
{\setlength{\emergencystretch}{3cm}%

\bibitem[\the\Quellennummer]{HTWgross}
HTW-Logo auf dem Deckblatt\par
\url{https://de.wikipedia.org/wiki/Datei:Logo_HTW_Berlin.svg} \par
 Stand: 17.08.2018 um 14:49 Uhr

\advance\Quellennummer by 1
 
\bibitem[\the\Quellennummer]{HTWklein}
HTW-Logo in der Kopfzeile\par
\url{http://tonkollektiv-htw.de/} \par
 Stand: 17.08.2018 um 14:53 Uhr

\advance\Quellennummer by 1

\bibitem[\the\Quellennummer]{SkriptSchulte}
Skript Automation in regenerativen Energiesystemen\par
Prof.\xspace Dr.\xspace -Ing.\xspace Horst Schulte

\advance\Quellennummer by 1

\bibitem[\the\Quellennummer]{LinBrandstaedter}
Anleitung Linearisierung eines zeitinvarianten,\par
nichtlinearen Zustandmodells\par
Prof.\xspace Dr.\xspace -Ing.\xspace Heide Brandstädter

\advance\Quellennummer by 1

\bibitem[\the\Quellennummer]{RegelungBuss}
Regelungs- und Steuerungstechnik: Polstellenverteilung\par
Prof.\xspace Dr.\xspace -Ing.\xspace M. Buss

\advance\Quellennummer by 1

}
\end{thebibliography}

\end{document}
