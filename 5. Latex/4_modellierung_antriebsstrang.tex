\section{Modell des Antriebsstranges} \label{modellierung_antriebsstrang}
% Aaron

Dieser Abschnitt befasst sich mit der Modellierung und Simulation des Antriebsstranges der Windturbine. Das Modell soll anschließend zusammen mit dem Modell des Turmes und des Blattes, sowie der Aerodynamik als Teilmodelle in Simulink zusammengeführt werden. \\
Der Antriebsstrang besteht dabei grundsätzlich aus einem Rotor (Blätter und Nabe), einem Getriebe, einem Synchrongenerator sowie einer Welle, welche alle Komponenten mechanisch verbindet. \autoref{fig:Bild4.1} zeigt einer Übersichtsgrafik der Modellparameter als Ein- \bzw Ausgänge des Teilmodells.

\begin{figure}[H]
   \centering
   \begin{pspicture}[showgrid=false](0,0)(10,5)
        \psframe(0,0)(10,5)
        % Eingänge
        \psline{->}(1,3.5)(3.5,3.5)
        \rput(1,3.9){\footnotesize \acs{MG}}
        \psline{->}(1,2.5)(3.5,2.5)
        \rput(1,2.9){\footnotesize \acs{MR}}
        % Modell
        \psframe[linecolor=black,fillcolor=lightGrey,fillstyle=solid](3.5,0.5)(6.5,4.5)
        \rput(5,3.5){\small Modell des}
        \rput(5,3){\small Antriebsstranges}
        % Ausgänge
        \psline{->}(6.5,3.5)(9,3.5)
        \rput(9,3.9){\footnotesize \acs{omegaG}}
        \psline{->}(6.5,2.5)(9,2.5)
        \rput(9,2.9){\footnotesize \acs{omegaR}}
    \end{pspicture}
   \caption[Übersicht Antriebsstrangteilmodell]{Blockdarstellung des Antriebsstrangteilmodells inklusive der Ein- und Ausgangsparameter}
   \label{fig:Bild4.1}
\end{figure} % Teilmodell Antriebsstrang mit Ein- und Ausgangsparametern

Zu erkennen sind das \ac{MG} und das \ac{MR} als Eingangsgrößen des Modells. Das \acl{MR} wird in der Gesamtsimulation später vom Modell der Aerodynamik bereitgestellt. Das \acl{MG} geht aus dem Generator-Umrichter-Modell hervor, in welchem auch die Regelung der Windkraftanlage umgesetzt ist. \\
Als Ausgänge benötigt werden die \ac{omegaG} und die \ac{omegaR}. Erstere ist später ein Eingangsparameter des Generator-Umrichter-Modells und wird insbesondere für die Berechnung des \acl{MG} im Momentenregler benötigt. Die \acl{omegaR} ist ebenfalls ein Eingang des Generator-Umrichter-Modells sowie des Modells für die Aerodynamik. 

\subsection{Modellierung des Antriebsstranges}

Zunächst soll das Modell für den Antriebsstrang entwickelt werden, bevor es anschließend simulativ losgelöst in Simulink getestet und verifiziert werden kann. \autoref{fig:Bild4.2} zeigt modellhaft die Struktur des Antriebsstranges mitsamt der wirkenden Momente. Auf Basis der Abbildung erfolgt anschließend die Modellierung des Teilsystems.

\begin{figure}[H]
   \centering
   \begin{pspicture}[showgrid=false](0,0)(14.6,5)
        \psframe(0,0)(14.6,5)
        
        % Generator
        \psline(0.5,2.5)(2.1,2.5)
        \psarc[linecolor=darkgrey]{<-}(2,2.5){1}{150}{210}
        \rput(1.2,3.3){\footnotesize \acs{MG}}
        
        \psellipse(2.1,2.5)(0.3,0.61)
        \psline(2.1,3.1)(3.6,3.1)
        \psline(2.1,1.9)(3.6,1.9)
        \psarc(3.08,2.5){0.8}{-49}{49}
        \rput(3,2.5){\footnotesize \acs{JG}}
        
        \psline(3.9,2.5)(5,2.5)
        \psarc[linecolor=darkgrey]{<-}(3.6,2.5){1}{-30}{30}
        \rput(4.4,3.3){\footnotesize \acs{phiG}}
        
        % Dämpferglied
        \psline(5,3.1)(5,1.9)
        \pscoil[coilwidth=0.3](5,3.1)(7.05,3.1)
        \psline(5,1.9)(5.8,1.9)
        \psline(5.8,2.2)(5.8,1.6)
        \psline(5.8,2.2)(6.5,2.2)
        \psline(5.8,1.6)(6.5,1.6)
        \psline(6,2.15)(6,1.65)
        \psline(6,1.9)(7.05,1.9)
        \psline(7.05,3.1)(7.05,1.9)
        \rput(6,3.7){\footnotesize \acs{ks}}
        \rput(6,1.2){\footnotesize \acs{ds}}
        
        % Getriebe
        \psline(7.05,2.53)(8.1,2.53)
        \psline(7.05,2.47)(8.1,2.47)
        \psarc[linecolor=darkgrey]{<-}(6.7,2.5){1}{-30}{30}
        \rput(7.5,3.3){\footnotesize \acs{phiRtilde}}
        
        \psframe(8.1,1.9)(9.8,3.1)
        \rput(8.9,2.5){\footnotesize \acs{ng}}
        
        % Rotor
        \psline(9.8,2.53)(11.1,2.53)
        \psline(9.8,2.47)(11.1,2.47)
        \psarc[linecolor=darkgrey]{<-}(9.4,2.5){1}{-30}{30}
        \rput(10.2,3.3){\footnotesize \acs{phiR}}
        
        \psellipse(11.1,2.5)(0.3,0.61)
        \psline(11.1,3.1)(12.6,3.1)
        \psline(11.1,1.9)(12.6,1.9)
        \psarc(12.08,2.5){0.8}{-49}{49}
        \rput(11.98,2.5){\footnotesize \acs{JR}}
        
        \psline(12.9,2.5)(14,2.5)
        \psarc[linecolor=darkgrey]{<-}(12.4,2.5){1}{-30}{30}
        \rput(13.2,3.3){\footnotesize \acs{MR}}
    \end{pspicture}
   \caption[Modelldarstellung des Antriebsstranges]{Modellhafte des Antriebsstrangteilmodelles inklusive der wirkenden Momente}
   \label{fig:Bild4.2}
\end{figure} % Modelldarstellung des Antriebsstranges inkl. Momente

Die Modellbildung erfolgt über die Momentenbilanzierung. Diese besagt, dass die Summe aller wirkenden Momente gleich Null ist. Zunächst lässt sich die Summe aller Momente allgemein berechnen zu

\begin{align}
    \sum M = J \cdot \ddot\varphi.
    \label{eq:Gleichung4.1}
\end{align}

Nachfolgend wird eine Bilanzgleichung für den Generator (high speed shaft) und den Rotor (slow speed shaft) aufgestellt. Erstere ist in \autoref{eq:Gleichung4.2} und Letzere in \autoref{eq:Gleichung4.3} dargestellt.

\begin{align}
    \label{eq:Gleichung4.2}
   \acs{JR} \cdot \ddot{\varphi}_{\mathrm{R}} + \tilde{M}_{d_s} + \tilde{M}_{k_s} - \acs{MR} &= 0 \\
   \label{eq:Gleichung4.3}
   \acs{JG} \cdot \ddot{\varphi}_{\mathrm{G}} - {M}_{d_s} - {M}_{k_s} + \acs{MG} &= 0
\end{align}

Bei $\tilde{M}_{d_s}$ handelt es sich um das Moment, welches sich aus dem \ac{ds} ergibt. Es wird berechnet zu

\begin{align}
   \tilde{M}_{d_s} = \acs{ds} \cdot \left( \dot\varphi_{\mathrm{G}} - \tilde{\dot\varphi}_{\mathrm{R}}\right).
   \label{eq:Gleichung4.4}
\end{align}

Bei $\tilde{M}_{k_s}$ wiederum handelt es sich um das Moment, welches sich aus dem \ac{ks} ergibt. Dieses wird berechnet zu

\begin{align}
   \tilde{M}_{k_s} = \acs{ks} \cdot \left( \varphi_{\mathrm{G}} - \tilde{\varphi}_{\mathrm{R}}\right).
   \label{eq:Gleichung4.5}
\end{align}

Da es sich bei beiden Momenten um bezogene Größen (bezogen auf die Schnelle Welle) handelt, müssen diese noch umgerechnet werden. Dies geschieht über die \ac{ng}. $M_{d_s}$ und $M_{k_s}$ folgen somit zu

\begin{align}
    \label{eq:Gleichung4.6}
   M_{d_s} &= \acs{ng} \cdot \tilde{M}_{d_s} \\
   \label{eq:Gleichung4.7}
   M_{k_s} &= \acs{ng} \cdot \tilde{M}_{k_s}.
\end{align}

Auch bei $\tilde{\dot\varphi}_{\mathrm{R}}$ muss das Getriebeübersetzungsverhältnis wie folgt berücksichtigt werden:

\begin{align}
    \tilde{\dot\varphi}_{\mathrm{R}} = \acs{ng} \cdot \acs{phiR}
    \label{eq:Gleichung4.8}
\end{align}

Somit kann der Antriebsstrang abschließend modelliert werden zu

\begin{equation}
    \label{eq:Gleichung4.9}
   \boxed{\acs{JR} \cdot \ddot{\varphi}_{\mathrm{R}} + \acs{ng} \cdot \acs{ds} \cdot \left( \dot\varphi_{\mathrm{G}} - \acs{ng} \cdot \tilde{\dot\varphi}_{\mathrm{R}}\right) + \acs{ng} \cdot \acs{ks} \cdot \left( \varphi_{\mathrm{G}} - \acs{ng} \cdot \tilde{\varphi}_{\mathrm{R}}\right) - \acs{MR} = 0}
\end{equation}

\begin{equation}
   \label{eq:Gleichung4.10}
   \boxed{\acs{JG} \cdot \ddot{\varphi}_{\mathrm{G}} - \acs{ds} \cdot \left( \dot\varphi_{\mathrm{G}} - \acs{ng} \cdot \tilde{\dot\varphi}_{\mathrm{R}}\right) - \acs{ks} \cdot \left( \varphi_{\mathrm{G}} - \acs{ng} \cdot \tilde{\varphi}_{\mathrm{G}}\right) - \acs{MG} = 0}.
\end{equation}

\subsection{Simulative Modellverifikation des Antriebsstranges}

In diesem Unterkapitel wird das zuvor entwickelte Modell in Simulink implementiert und getestet. Auf eine Vollständige Verifikation muss an dieser Stelle jedoch verzichtet werden, da diese nicht losgelöst von den anderen Teilmodellen stattfinden kann. Geprüft wird jedoch, inwiefern einer Torsion der Welle Auftritt, wenn sich das \acl{MG} \bzw \acl{MR} sprunghaft ändert.

% Bild aus Simulink zur Blockstruktur

% Bild aus Matlab mit den geplotteten Ergebniswerten zur Wellentorsion (die findest du doch eh nicht hehe)